\documentclass[10pt,a4paper]{report}
\usepackage[italian]{babel}
\usepackage[utf8]{inputenc}
\usepackage[T1]{fontenc}
\usepackage{amsmath}
\usepackage{amsfonts}
\usepackage{amssymb}
\usepackage{graphicx}
\usepackage[width=18.00cm, height=25.00cm]{geometry}
\usepackage{lipsum}
\usepackage{ragged2e}
\reversemarginpar
\usepackage{marginnote}
\usepackage{paracol}
\usepackage{array}
\usepackage{hyperref}

\pagestyle{empty}
\setcolumnwidth{5.3cm}

\newcommand\headerstyle{\bfseries}

\newcolumntype{R}[1]{>{\raggedleft\let\newline\\\arraybackslash\hspace{0pt}}m{#1}}

\newcommand{\field}[5]{\vspace{2mm} \begin{paracol}{2}
        \noindent
        #1 \\ \textit{#2}
        \switchcolumn \noindent
        {\bfseries\Large #3} \\
        {\bfseries #4} \vspace{1mm} \\ \noindent
        #5
\end{paracol} \vspace{3mm}}

\newcommand{\progetto}[3]{\vspace{2mm} \begin{paracol}{2}
        \noindent
        \textbf{#1} \\ \textit{#2}
        \switchcolumn \noindent#3
\end{paracol}}


\newenvironment{topheader} {
    \begin{tabular}{R{5cm} l}
    {\bfseries \LARGE \name} & {\bfseries \LARGE \surname} \\
} {
    \end{tabular}
}

\newcommand\header[1]{\vspace{5mm}\noindent \rule{\textwidth}{1pt} \\ {\bfseries \MakeUppercase{#1}} \vspace{3mm}\\}

\newenvironment{twocolentry}[1]{
    \begin{paracol}{2}
        #1
        \switchcolumn\noindent
} {
    \end{paracol} \vspace{3mm}
}

\newenvironment{study}[4]{
    \begin{twocolentry}
        {\noindent #3 \\ {\itshape #4}}
        {\noindent\bfseries\Large#1} \\
        {\bfseries #2} \vspace{2mm} \\
} {
    \end{twocolentry}
}

\newenvironment{work}[4]{
    \begin{twocolentry}
        {\noindent #3 \\ {\itshape #4}}
        {\noindent\bfseries\Large#1} \\
        {\bfseries #2} \vspace{2mm} \\
} {
    \end{twocolentry}
}

\newenvironment{project}[2]{
    \begin{twocolentry}
        {\noindent {\bfseries#1} \\ {\itshape#2}}
} {
    \end{twocolentry}
}

\newenvironment{competence}[1]{
    \begin{twocolentry}
        {\noindent {\bfseries#1}}
} {
    \end{twocolentry}
}


% \usepackage[defaultfam,light,tabular,lining]{montserrat}
% \renewcommand*\oldstylenums[1]{{\fontfamily{Montserrat-TOsF}\selectfont #1}}
% \allowdisplaybreaks

\usepackage[osf,sc]{mathpazo}
\usepackage[scaled=0.90]{helvet}
\usepackage[scaled=0.85]{beramono}

\newif\ifcomplete
\completetrue
% \completefalse


\begin{document}

    \def\name{Matteo}
    \def\surname{Dalle Vedove}
    \makeatletter
    \begin{topheader}
        \ifcomplete e-mail & matteodv99tn@gmail.com \\ \fi
        \ifcomplete indirizzo & via San Biagio, n. 14, Borghetto all'Adige (TN), CAP 38063, Italia \\ \fi
        \ifcomplete telefono & (+39) 349 843 9679 \\ \fi
        \ifcomplete nascita & 12 maggio 1999, Rovereto \\ \fi
        \ifcomplete nazionalità & italiana \\ \fi
        \ifcomplete codice fiscale & DLLMTT99E12H612K \\ \fi
        LinkedIn & \href{https://www.linkedin.com/in/matteo-dalle-vedove-71a290239/}{\textit{Matteo Dalle Vedove}} \\
        GitHub & \href{https://github.com/matteodv99tn}{\textit{matteodv99tn}} \\
    \end{topheader}

    \header{Istruzione}

    \begin{study}
        {Dottorato in \textit{Robotics and Intelligent Machines}}
        {Università degli Studi di Trento}
        {Trento}
        {2023-2026}
        Percorso di dottorato di interesse nazionale in \textit{Robotics and Intelligent Machines}, curricola \textit{Industry 4.0}, con sede amministrativa presso l'universita' di Genova e sede operativa presso l'universita' di Trento. \\
        Supervisori: Fontanelli Daniele e Saveriano Matteo.
    \end{study}

    \begin{study}
        {Ingegneria Meccatronica}
        {Università degli Studi di Trento}
        {Trento}
        {2021-2023}
        Laurea magistrale in \textit{Mechatronics Engineering}, curricola \textit{Electronics and Robotics}, del Dipartimento di Ingegneria Industriale.

        Maturati particolari interessi nell'ambito dell'identificazione di sistemi, controllo, programmazione, metodi numerici e simulazione.

        Votazione di uscita: 110 con lode. Titolo della tesi: \textit{Analysis and Programming of Delta Robot for Automatic Waste Sorting}.
    \end{study}

    \begin{study}
        {Ingegneria Industriale}
        {Università degli Studi di Trento}
        {Trento}
        {2018-2021}
        Laurea triennale in \textit{Ingegneria Industriale}, curricola \textit{metodologico meccatronica}. del Dipartimento di Ingegneria Industriale.

        Votazione di uscita: 102 su 110. Titolo della tesi: \textit{Analisi di circuiti ad applicazione digitale in tecnologia c-MOS}.
    \end{study}

    \begin{study}
        {Liceo Scientifico - opzione Scienze Applicate}
        {Liceo Antonio Rosmini}
        {Rovereto, TN}
        {2012-2018}
        Votazione di uscita: 80 su 100.
    \end{study}


    \header{Esperienze lavorative}
    \begin{work}
        {Tirocinio finalizzato alla stesura della tesi}
        {BM Group - Polytec}
        {Condino, TN}
        {luglio-ottobre 2023}
        Principali attività svolte nel settore della programmazione PLC (principalmente \textit{structured text} e \textit{ladder diagrams}) nell'ambiente di sviluppo \textit{Codesys} per la gestione della logica di una stazione robotizzata e la pianificazione dei movimenti di un robot delta per la cernita di rifiuti nel settore del riciclo del metallo.
    \end{work}

    \begin{work}
        {Collaboratore Didattico}
        {Fondazione Museo Civico di Rovereto}
        {Rovereto, TN}
        {2019-2023}
        Insegnante di robotica per ragazzi dai 6 ai 15 anni. Principali attività svolte e competenze acquisite:
        \begin{itemize}
            \item insegnamento del linguaggio di programmazione a blocchi con software proprietari per robot \textit{LEGO}, modelli \textit{Mindstorm EV3} e \textit{SPIKE Prime}.
        \end{itemize}
        Partecipazione alla competizione internazionale \textit{First Lego League} come arbitro nazionale:
        \begin{itemize}
            \item organizzazione e gestione di competizioni nazionali di robotica;
            \item valutazione delle performance di squadre di ragazzi dagli 8 ai 16 anni.
        \end{itemize}
    \end{work}

    \header{Progetti}
    \begin{project}
        {Industrial AI Challenge}
        {2022}
        Competizione universitaria di squadra organizzata da \textit{Hub Innovazione Trentino} nell'ambito dell'intelligenza artificiale in scopi industriali. \\
        La sfida, proposta dall'azienda \textit{Melinda}, è quella di ottimizzare il processo di conservazione delle mele in modo da diminuire i costi legati all'energia.

        Nella squadra il mio compito principale è stato quello di effettuare system identification in modo da ricreare un ambiente virtuale per l'allenamento di un agente di reinforcement learning.
    \end{project}

    \begin{project}
        {ProtoChallenge}
        {2020}
        Competizione universitaria di squadra organizzata da \textit{Hub Innovazione Trentino} nell'ambito dell'additive manufacturing. \\
        La sfida, proposta dall'azienda \textit{PAMA}, è quella di ottimizzare topologicamente una testa universale che fornisce due assi rotazionali di posizionamento per una alesatrice industriale. Le principali competenze acquisite sono:
        \begin{itemize}
            \item apprendimento del software \textit{NTopology} per lo studio e ottimizzazione topologica di componenti meccanici;
            \item programmi per l'analisi ad elementi finiti FEM.
        \end{itemize}
        Obiettivi raggiunti:
        \begin{itemize}
            \item riduzione della massa del pezzo del 45\%, mantenendo una rigidezza complessiva nel range prefissato $\pm$10\%;
            \item miglioramento della geometrica ai fini di aumentare la dissipazione di calore;
            \item stimata una riduzione del 60-70\% dei tempi di produzione del componente.
        \end{itemize}
    \end{project}

    \begin{project}
        {First Lego League}
        {2014-2018}
        Competizione internazionale di robotica, partecipazione con una squadra scolastica. Nelle prime due edizioni partecipazione attiva alla gara come membro, gli anni successivi come mentore delle squadre più giovani; principali competenze acquisite:
        \begin{itemize}
            \item costruzione e programmazione di robot \textit{LEGO Mindstorm EV3};
            \item redazione di progetti scientifici inerenti a tematiche di rilievo, proposte a tutte le squadre da parte dell'organizzazione.
        \end{itemize}
        Riconoscimenti ricevuti:
        \begin{itemize}
            \item terzo posto a livello nazionale, edizione FLL2016;
            \item premio internazionale \textit{Out of the Box} alla sfida \textit{Asia Open Pacific}, tenutasi a Sydney, edizione FLL2017;
            \item premio \textit{Oltre la Robotica} insignito dal Ministero Italiano dell'Istruzione, edizione FLL2017.
        \end{itemize}
    \end{project}

    \header{COMPETENZE}
    \begin{competence}{Programmazione}
        \begin{itemize}
            \item programmazione avanzata in linguaggio \textit{C} e \textit{C++}, con particolare attenzione allo sviluppo di algoritmi numerici e sistemi embedded. Conoscenza dei paradigmi di programmazione ad oggetti. Esperienza di sviluppo di interfacce hardware e controllori nel framework di \textit{ROS2};
            \item programmazione avanzata in \textit{Python};
            \item conoscenza medio-avanzata del linguaggio \textit{Matlab};
            \item conoscenza discreta di calcolo simbolico con software \textit{Maple} e \textit{Mathematica};
            \item conoscenza base di scripting in \textit{bash};
            \item conoscenza discreta di programmazione PLC in ambiente \textit{Codesys};
            \item buona conoscenza di redazione di documenti \LaTeX.
        \end{itemize}
    \end{competence}

    \begin{competence}{Lingue}
        \begin{itemize}
            \item italiano: madrelingua;
            \item inglese: certificazione \textit{B2 specialistico per ingegneria} sostenuto presso il \textit{CLA} dell'Università di Trento; buon livello di comprensione, comunicazione orale e scrittura, soprattutto nell'ambito di discorsi ingegneristici;
            \item spagnolo: base.
        \end{itemize}
    \end{competence}

    \header{VOLONTARIATO}
    \begin{project}
        {Banda Sociale di Ala}
        {2014-attuale}
        Suonatore di trombone ed euphonium all'interno della banda sociale comunale. Dal 2019 membro del consiglio direttivo dell'associazione.
    \end{project}

    \begin{project}
        {Associazione MindsHub}
        {2015-2019}
        Mentor dell'area robotica dell'associazione, in particolare per quanto riguarda l'introduzione alla costruzione e programmazione a blocchi di robot per ragazzi della scuola primaria.
    \end{project}


    \vspace{5mm}

    \noindent
    \textit{Il sottoscritto, consapevole che – ai sensi dell’art. 76 del D.P.R. 445/2000 – le dichiarazioni mendaci, la
    falsità negli atti e l’uso di atti falsi sono puniti ai sensi del codice penale e delle leggi speciali, dichiara che le
    informazioni rispondono a verità. \vspace{2mm} \\
    Il sottoscritto in merito al trattamento dei dati personali esprime il proprio consenso al trattamento degli
    stessi nel rispetto delle finalità e modalità di cui al d.lgs. n. 196/2003.}

    \vspace{4mm}
    \begin{flushright}
        Matteo Dalle Vedove \\
        Trento, \today \\
        \ifcomplete \includegraphics[width=5cm]{signature} \fi
    \end{flushright}



\end{document}
